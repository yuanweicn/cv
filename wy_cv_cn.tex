\documentclass[letterpaper]{article}

\usepackage{CJKutf8}
\usepackage{hyperref}
\usepackage{geometry}
\usepackage{enumitem}
\usepackage{etaremune}
\usepackage{amssymb}
\usepackage[T1]{fontenc}
\usepackage[sc,osf]{mathpazo}
\begin{CJK}{UTF8}{gkai}
% Set your name here
\def\name{袁巍}
\def\ename{Yuan Wei}

%math commands
\newcommand{\R}{\mathbb R}

% Replace this with a link to your CV if you like, or set it empty
% (as in \def\footerlink{}) to remove the link in the footer:
% \def\footerlink{http://jblevins.org/projects/cv-template/}

% The following metadata will show up in the PDF properties
\hypersetup{
  colorlinks = true,
  urlcolor = black,
  pdfauthor = {\ename},
  pdfkeywords = {mathematics},
  pdftitle = {\ename: Curriculum Vitae},
  pdfsubject = {Curriculum Vitae},
  pdfpagemode = UseNone
}

\geometry{
  body={6.5in, 8.5in},
  left=1.0in,
  top=1.25in
}

% Customize page headers
\pagestyle{myheadings}
\markright{\name}
\thispagestyle{empty}

% Custom section fonts
\usepackage{sectsty}
\sectionfont{\rmfamily\mdseries\Large}
\subsectionfont{\rmfamily\mdseries\itshape\large}

% Other possible font commands include:
% \ttfamily for teletype,
% \sffamily for sans serif,
% \bfseries for bold,
% \scshape for small caps,
% \normalsize, \large, \Large, \LARGE sizes.

% Don't indent paragraphs.
\setlength\parindent{0em}

% Make lists without bullets
\renewenvironment{itemize}{
  \begin{list}{}{
    \setlength{\leftmargin}{1.5em}
  }
}{
  \end{list}
}

\begin{document}
% Place name at left
{\huge \name}

% Alternatively, print name centered and bold:
%\centerline{\huge \bf \name}

\vspace{0.25in}

\begin{minipage}{0.60\linewidth}
  \href{http://www.math.ac.cn/}{数学所}\\ 
  \href{http://www.math.ac.cn/}{中国科学院数学与系统科学研究院} 
  \\ 中关村东路55号,北京,中国 (100190)
\end{minipage}
\begin{minipage}{0.4\linewidth}
  \begin{tabular}{ll}
    办公室电话: & (86-10) 6255-3362 \\
    移动电话: & (86) 13691205628 \\
    电子邮件: & \href{mailto:wyuan@math.ac.cn}{\tt wyuan@math.ac.cn} \\
  \end{tabular}
\end{minipage}

\section*{教育经历}

\begin{description}[leftmargin=0.95in, labelwidth=0.9in]
  \item[2003-2009] 中国科学院数学与系统科学研究院,获理学博士学位。\\
                   论文题目:Kadison-Singer代数\\
                   导师:葛力明
  \item[2006-2009] 新罕布什尔大学,获理学博士学位。\\
                   论文题目:Kadison-Singer代数\\
                   导师:葛力明
   \item[1999-2003] 中国科学技术大学,获学士学位。
\end{description}

\section*{工作经历}
\begin{description}[labelwidth=0.9in]
  \item[2009-今] 中国科学院数学与系统科学研究院,助理研究员。
  \item[2006-2008] 新罕布什尔大学,助教。
\end{description}

\section*{研究方向}
算子代数及相关领域

\section*{研究工作}

\subsection*{发表文章}

\begin{etaremune}
 \item W. Wu and W. Yuan, A remark on central sequence algebras of the tensor product of II$_1$ factors, Proc. Amer. Math. Soc. to appear.

 
 \item W. Wu and W. Yuan, On generators of abelian Kadison-Singer algebras in matrix algebras, Linear Algebra and its Applications, Vol 440, 197-205, 2014.

 \item A. Dong, W. Wu and W. Yuan, On small subspace lattices in Hilbert space, Journal of the Australian Math Society, published online 2013.

  \item A. Dong, W. Yuan, C. Hou and G. Chen, Representations and operations on reflexive subspace lattices, Scientia Sinica Mathematica,
         42(4), 321-328, 2012.
   \item C.Hou and W.Yuan, Minimal generating reflexive lattices of projections in finite von Neumann algebras, Math. Ann.
         Volume 353, Issue 2, 499-517, June 2012.
   \item L.Wang and W.Yuan, A new class of Kadison-Singer algebras, Expo. Math., 29, 126-132, 2011.
   \item W.Wu and W.Yuan, The crossed product von Neumann algebras associated with SL$_2(\R)$, Taiwanese Journal of Mathematics, 14(4), 1501-1515, 2010.
   \item L.Ge and  W.Yuan, Kadison-Singer Algebras, II: General Case, Proc.Nat.Acad.Sci. U.S.A., Vol 107,no.11, 4840-4844, February 25, 2010.
   \item L.Ge and  W.Yuan, Kadison-Singer Algebras, Hyperfinite Case, Proc.Nat.Acad.Sci. U.S.A., Vol.107, no.5, 1838-1843, February 2, 2010.
   \item W.Wu and W.Yuan, A Note on the Crossed Product of von Neumann Algebras, Acta Mathematica Sinica Chinese Series, 51(4) 803-808, 2008.
\end{etaremune}

   
\subsection*{未完成}
\begin{itemize}
\item B. Fu, L. Ge and W. Yuan, On Sarnak's Conjecture.
\item L. Ge, W. Wu and W. Yuan, On the sphere invariant automorphisms of
finite von Neumann algebras.
\item W. Wu and W. Yuan, On the unbounded operators with trivial
bounded commutant in II$_1$ factors.
\end{itemize}

\section*{大会报告}
\begin{itemize}
  \item Kadison-Singer Algebras, Operator Algebras and Related Topics, 北京,2010年7月24日。
\end{itemize}

\section*{邀请报告}
\begin{itemize}
  \item Kadison-Singer Algebras
  \begin{itemize}
  \item 曲阜师范大学,数学系,2009年12月28日。
  \item 新加坡国立大学,数学系,2009年5月2日。
  \end{itemize}
 \item On a new class of non selfadjoint algebras
  \begin{itemize}
  \item 重庆大学,数学系,2011年9月21日。
  \end{itemize}
  \item Unbounded Operator With Trivial Relative Commutant
     \begin{itemize}
         \item Dartmouth College, 2013年11月2日。
     \end{itemize}
\end{itemize}

\section*{教学}
\begin{itemize}
\item Calculus II (426), University of New Hampshire, Autumn, 2013.
\item 泛函分析 II,中国科学院大学,春季,2011, 2012, 2013。
\item 自由概率论简介(小型课程),华东理工大学,2009年六月。
\end{itemize}


\section*{讨论班}
\begin{itemize}
  \item 算子代数研讨班(王利广), 曲阜师范大学,2013年7月。
  \item 算子代数与数论研讨班,数学与系统科学研究院,2012年四月-2013年1月。
  \item 算子代数研讨班(吴文明),重庆师范大学,2011年10月。
  \item 算子代数研讨班(葛力明),晨兴数学中心,2010,2011。
  \end{itemize}

\section*{其它}
担任如下期刊审稿人: 
\begin{itemize}
  \item Expositiones Mathematicae
  \item Science China Mathematics 
  \item Acta Mathematica Sinica 
  \item Acta Mathematicae Applicatae Sinica
\end{itemize}

\section*{所获奖项}
\begin{itemize}
  \item 钟家庆数学奖, 2009。
\end{itemize}


\bigskip
% Footer
%\begin{center}
%  \begin{footnotesize}
%    %Last updated: \today \\
%    \href{\footerlink}{\texttt{\footerlink}}
%  \end{footnotesize}
% \end{center}


\end{document}
\end{CJK}
